\documentclass[journal]{IEEEtrancz}
\usepackage[utf8]{inputenc}
\usepackage[czech]{babel}
\usepackage{graphicx}
\usepackage{mathtools}

\begin{document}

\title{TSP s několika cestujícími}
\author{Filip Klimeš}

\maketitle

\begin{abstrakt}
% Zde uvést krátky odstavec s abstraktem. V abstraktu stručně shrnuto o čem článek je.
% Čtenář je motivován, proč by měl článek číst, poslední věta zpravidla.
% shrnuje dosažené výsledky.

\end{abstrakt}

\IEEEpeerreviewmaketitle

\section{Zadání}
% Několika větami naformulované zadání práce tak, jak jste je dohodli se cvičicím.
Za úkol máme v neorientovaném grafu o N uzlech nalézt množinu cest pro M cestujících.
Každá cesta vychází z depotu D a na konci se do něj vrací. Všechny uzly kromě depotu
jsou navštíveny právě jednou, žádná z cest nesmí mít nulovou délku. Cílem je minimalizovat
délku nejdelší trasy.

\section{Úvod}
% Každý článek obsahuje krátký úvod do problematiky, v našem případě do problému.
% zkoumaných dat. Úvod by neměl obsahovat výklad funkcionality použité neuronové sítě.
Problém obchodního cestujícího (TSP) je známý NP-těžký problém. Jde o nalezení nejkratší
cesty v grafu G, která navštíví všechny jeho uzly. Matematicky řečeno, jedná se o nalezení
nejkratší hamiltonovské kružnice v úplném ohodnoceném grafu.

TSP s několika cestujícími, také známé jako Vehicle Routing Problem (VRP), je obecnějším
případem TSP. Lze ukázat, že VRP je NP-těžký problém. Pokud N $\neq$ NP, nelze VRP řešit
v polynomiálním čase. Aplikování evolučních algoritmů je tedy na místě.

\section{Cíl práce}
% Naučte se formulovat cíl práce, zde podrobně rozeberte zadání, v závěru pak zhodnotíte, zdali
% bylo cíle dosaženo.

Cílem této práce je navrhnout, implementovat a změřit kvalitu

\begin{itemize}
\item lokálního prohledávání
\item jednoduchého evolučního algoritmu
\item hybridizovaného evolučního a memetického algoritmu
\end{itemize}

\section{Popis reprezentace problému}

Předpokládáme, že města jsou očíslovaná od 1 do N.
Problém budeme reprezentovat jako posloupnost čísel označujících
daná města. Mezi tuto posloupnost bude vloženo M - 1
oddělovačů, a to tak, že posloupnost nesmí začínat ani končit
oddělovačem a mezi dvěma oddělovači musí být alespoň jedno číslo.
Posloupnost čísel mezi dvěma oddělovači reprezentuje jednu z cest.
Reprezentace zajišťuje, že vždy bude existovat M cest a každý
cestující vždy navštíví alespoň jedno město.

$$
c_1 c_2 ... c_k S_1 c_k+1 ... c_l S_2 c_l+1 ... S_M-1 c_m+1 ... c_N
$$

\section{Popis fitness funkce}

Fitness funkce spočte euklidovskou délku jednotlivých cest a vybere
tu nejvyšší z nich.

$$
fitness(Individual) = max(\foreach p \belongs P ||p||)
$$

\section{Popis algoritmů}



\subsection{Popis operátorů}

\subsubsection{Perturbační operátor u lokálního prohledávání}

% TODO: taboo search
% TODO: swap mutace

\subsubsection{Operátory křížení a mutace u EA}

Operátor křížení využívá techniky edge-recombination. Operátor nejprve
náhodně zvolí oddělovače jednoho z rodičů – ty budou složit jako oddělovače
potomka. Následně si operátor z obou rodičů sestaví tzv. adjacency list, tedy
seznam sousedností jednotlivých měst. Tzn. pokud v jednom z rodičů vedla cesta
z města A do města B, potom seznam sousedností bude pro město A obsahovat město B
a vice versa. Začátek reprezentace je zvolen náhodně jako první město jednoho z rodičů.
Nakonec následuje algoritmus pro edge-recombination. Dokud neobsahuje potomek N měst,
je podle seznamu sousedností náhodně vybíráno následující město, které ještě není v
potomkovi zastoupeno.

% TODO: pseudokód

Operátor mutace zvolí dvě náhodná města (může zvolit i oddělovače) a ty
mezi sebou vymění. Pokud výsledná reprezentace není validní, tj. pokud reprezentace
začíná nebo končí oddělovačem nebo jsou dva oddělovače vedle sebe, jsou znovu zvolena
a vyměněna další dvě města, dokud reprezetace není validní.

\subsubsection{Způsob hybridizace EA a lokálního prohledávače u memetického algoritmu}

Pro hybridizaci EA zvolíme heuristiku známou jako 3-opt algoritmus. Po dokončení
crossoveru a mutace přidáme krok memetiky, teda individuálního učení, který nastane
u jedince s předem zvolenou pravděpodobností. Algoritmus 3-opt postupně vybírá kombinace
3 uzlů v nejdelší cestě jedince a zkouší rekombinovat hrany, kterými jsou spojeny. Pokud některá
z kombinací zmenšuje délku cesty, je tato rekombinace uložena.

\subsection{Hodnoty parametrů}

% TODO: pravděpodobnost crossoveru
% TODO: pravděpodobnost mutace
% TODO: pravděpodobnost memetiky


\section{Experimenty}
Tato část popisuje, jak byly experimenty provedeny, nebojte se použít obrázků a tabulek, které
zvýší vypovídací schopnost textu a přehlednost.

\begin{figure}[ht]
  \centering
    \includegraphics{figure}
      \caption{Název a stručný popis obrázku}
    \label{fig:exfig}
\end{figure}

Do experimetální části se přímo hodí obrázky. Každý obrázek musí být řádně vysvětlen
a okomentován. U grafů musí být popsány osy, pod obrázek umístíme název a
krátké vysvětlení toho, co na obrázku je. Obrázek \ref{fig:exfig} je příklad
umístění obrázku do dokumentu.

\begin{table}
  \centering
  \caption{Parametry experimentu}
  \begin{tabular}{|l||c|c|c|}
  \hline
    & A & B & C \\
  \hline
  \hline
  S učitelem  & 0.22 & 0.27 & 0.29 \\
  \hline
  Bez učitele & 0.12 & 0.17 & 0.20 \\
  \hline
  \end{tabular}
  \label{tab:extab}
\end{table}

Výsledky experimentu je vhodné shrnout tabulkou, příkladem
je tabulka \ref{tab:extab}. Pro tabulku platí totéž co
pro obrázek s výjimkou toho, že popis a název tabulky je nad tabulkou. Z tabulky
musí být jasné, co je jejím obsahem. Vysvětlení obsahu je vhodné uvést
do textu poblíž tabulky.

\section{Diskuse}
Referát na tento předmět by neměl být větší než 3 strany v tomto formátu.
Vlastní text by měl obsahovat úvod do problematiky zkoumaných dat, jejich rozbor. Návrh
použité neuronové sítě a v tabulce přehledné vstupní nastavení experimentu. Vždy je vhodné
uvést příklad zkoumaných dat, jaké obsahují atributy apod. Diskuse je důležitá kapitola,
která rozebírá dosažené výsledky.

\section{Závěr}
Každý článek/referát musí obsahovat závěr, který stručně shrnuje dosažené výsledky
experimentů. Závěr neobsahuje žádná nová zjištění, která by předtím v textu nebyla rozvedena.
Závěr je nejdůležitější část článku. Na základě závěru zpravidla pokrařujeme ve čtení
publikace.

\section*{Poděkování}
Volitelně poděkování, třeba spolupracující instituci.

\begin{literatura}{1}

\bibitem{autor:rok}
Autor P., Autor M.: \emph{Název knihy}, druh literatury, 2002.

\end{literatura}
\end{document}
